\documentclass[]{report}   % list options between brackets
\usepackage{}              % list packages between braces

% type user-defined commands here

\begin{document}

\title{Website/Blog with Pelican}   % type title between braces
\author{Conor Gilmer}         % type author(s) between braces
\date{January 31, 2014}    % type date between braces
\maketitle

\begin{abstract}
  A brief introduction to using Pelican for writing blogs and websites.
\end{abstract}

\chapter{Installation}             % chapter 1
\section{Introduction}     % section 1.1
\subsection{History}       % subsection 1.1.1

\chapter{Building}           % chapter 2
\section{Add Content}
Add content in the content directory, any directory set up in the content folder will be a blog, a specific directory here will be pages, and here is where static pages go (if you want a website without any blog/news items and only pages, all you will have here is a pages folder.
Pelican can use a number of formatting languages, Markdown is the one I favour to use here. 
\section{Running pelican}     % section 2.1


\subsection{Testing local server}         % subsection 2.1.1


\chapter{Publish}           % chapter 3
\section{Deploying}
One the static website has been generated by Pelican it can then be deployed to a host. In generating the website the domain/URL and path would have been specified in the configuration file.
\subsection{github pages}
Static site generators are great for publishing to \it{github pages}.
Where your site would be hosted on github, and to publish it to it you would use github

set up a github account (www.github.com)
set up a github pages repository


git add .
git commit -m "Update Blog"
git push origin master

username.github.io

\subsection{FTP to Host}
\subsection{Dropbox}

\begin{thebibliography}{9}
  % type bibliography here
\end{thebibliography}


\end{document}
