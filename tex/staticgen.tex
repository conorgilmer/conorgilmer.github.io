\documentclass[12pt]{article}			% For LaTeX 2e
						% other documentclass options:
						% draft, fleqn, openbib, 12pt
\usepackage{graphicx}	 			% insert PostScript figures
%% \usepackage{setspace}   % controllabel line spacing
%% If an increased spacing different from one-and-a-half or double spacing is
%% required then the spacing environment can be used.  The spacing environment 
%% takes one argument which is the baselinestretch to use,
%%         e.g., \begin{spacing}{2.5}  ...  \end{spacing}


% the following produces 1 inch margins all around with no header or footer
\topmargin	=10.mm		% beyond 25.mm
\oddsidemargin	=0.mm		% beyond 25.mm
\evensidemargin	=0.mm		% beyond 25.mm
\headheight	=0.mm
\headsep	=0.mm
\textheight	=220.mm
\textwidth	=165.mm
					% SOME USEFUL OPTIONS:
% \pagestyle{empty}			% no page numbers
 \parindent  15.mm			% indent paragraph by this much
 \parskip     2.mm			% space between paragraphs
% \mathindent 20.mm			% indent math equations by this much

\newcommand{\MyTabs}{ \hspace*{25.mm} \= \hspace*{25.mm} \= \hspace*{25.mm} \= \hspace*{25.mm} \= \hspace*{25.mm} \= \hspace*{25.mm} \kill }

\graphicspath{{../Figures/}{../data/:}}  % post-script figures here or in /.

					% Helps LaTeX put figures where YOU want
 \renewcommand{\topfraction}{0.9}	% 90% of page top can be a float
 \renewcommand{\bottomfraction}{0.9}	% 90% of page bottom can be a float
 \renewcommand{\textfraction}{0.1}	% only 10% of page must to be text

\alph{footnote}				% make title footnotes alpha-numeric

\title{Setting up a Website/\ Blog\\~\\ using Static site generators\\~\\}	% the document title

\author{	Conor Gilmer BSc\\	% author information
		Dublin \\
		Ireland \\~\\
		}

\date{\today}				% your own text, a date, or \today

% --------------------- end of the preamble ---------------------------

%\documentclass{article}
%\usepackage{hyperref}
%\hypersetup{colorlinks, %
%	citecolor=black,%
%	linkcolor=black,%
%	urlcolor=black,%
%	pdftex}	
%\usepackage{graphicx}

%\title{ Road Trip\\ A Journey to Cork For Noel Mahers Bachelor Party\\ The procedings shall take the form of a Road trip\\}
%\author{Conor Gilmer $<$\href{mailto:cgilmer@tinet.ie}{cgilmer@tinet.ie}$>$}


\begin{document}
\maketitle
\newpage
\tableofcontents
\newpage

\section{Static website generators}
Recently I have begun to look again at static website generators for creating websites and blogs. Static Website Generation is not a new concept, I have used a number of Perl based ones and eve LaTex in the past, but more recently they have seemed to get some energy as alternatives to a Content Management System (CMS) and particularly as they suit documentation and publishing to some cloud services such as \textit{github pages}. There is a number of them about and I am going to outline my thoughts and how to use them.
\begin{itemize}
\item Jeykll
\item Hyde
\item Pelican
\item Nikola
\end{itemize}


\newpage
\section{Jeykll}
Is a Ruby Static website generator, it seems quite favoured with github pages
No specific ruby coding skills are needed, but ruby is needed.

\newpage
\section{Hyde}
Is a Static website generator which uses python, similar to its alter ego Jeykll.
While it uses python, no specific python knowledge is required to use it. (In my case I used Python 2.7 installed with Anaconda) Hyde uses a Django based templating system, Jinja2. 

\subsection{Install}
The easiest way to install Hyde was to use \textit{easy\_install} or \textit{pip install}.
\begin{verbatim}
sudo easy_install hyde
or
pip install hyde
pip install jinja
\end{verbatim}

\subsection{Create Hyde Project}
\begin{verbatim}
mkdir hydesite
cd hydesite
hyde -s create

\end{verbatim}

\subsection{Generate Site and Run on Local Server}
You then generate the static html, and you can view it by running the hyde server.
\begin{verbatim}
hyde gen
hyde serve
\end{verbatim}

\newpage
\section{Pelican}
Is a Static website generator which uses python, although again no specific python skills are required to use the system.
\subsection{Installation}
\begin{verbatim}
pip install pelican
pip install Markdown
mkdir mysite
cd mysite
pelican-quickstart
\end{verbatim}
This uses \textit{pip} to install pelican and markdown, \textit{pelican-quickstart} sets up the plumbing of your pelican website.\par
The site is configured using settings.py (or pelicanconf.py).
The layout is setout in the theme which is used, at the moment we will use one of the simple themes notforme. You can change a theme or modify it or compose your own by editing the css and templates for the theme.
\subsection{Adding Content}
Here we will have two types of content on the website standard pages, and blog posts.
\subsubsection{Pages}
Pages are stored in the pages directory in the content directory e.g. about.md, contact.md etc. they are traditional webpages and are not listed or timestamped with author or category information as Blog post or news item is.

\begin{verbatim}
Title: About

##This is About page
This is a blog generated by Pelican using Python...
\end{verbatim}
(if you were just using it to generate a static website without any news items or blog you would just have the pages of the site here.

\subsubsection{Blogs}
Any markdown page in the content directory which is in a directory other than pages is treated as a blog.

\begin{verbatim}
Date: 2015-1-9
Title: My First Blog Post

##this is my first blog post in pelican##
Here is the main text of my first blog post

\end{verbatim}
The Date is the time the post will timestamped at, and you can have minutes as well, the Title is the title you will see and the
\subsection{Generating the Website}
\begin{verbatim}
pelican content/ -s settings.py
\end{verbatim}
The site is then generated in the \textit{output} directory.

\newpage
\section{Deploying/Publishing}
\subsection{Github Pages}
Most of the newer static website generators are convienently setup for publishing ong github pages (username.github.io).
\subsection{DropBox}
You can also deploy to your Dropbox account.
\subsection{FTP to a Host}
You can just FTP the contents of the output directory to you host, the website should be generated to the domain and page of your host.

\begin{verbatim}
git add .
git commit -m "updates"
git push origin master
\end{verbatim}

\newpage
\section{Nikola}
Another static website generator.

\newpage
\section{Markdown}
Since i have used Markdown for the above here is a few items from it

\subsection{Headings}
Headings are indicated by the Hash Symbol, the number of which indicate if it is H1, H2, H3 etc.
\begin{verbatim}
#This is H1
##This is H2
###This is H3
####This is H4
#####This is H5
######This is H6
\end{verbatim}
\subsection{Bold, Italics or Strikethrough}
For empasis Bold or Italics are often used, and sometime text needs to have a line through it.\par
Italics uses *asterisks* or \textunderscore underscores\textunderscore resulting in \textit{italicised} \par
Bold uses **two asterisks** or \textunderscore\textunderscore two underscores\textunderscore\textunderscore resulting in \textbf{bolded} \par


Strikerhough is achived by using two tildes(\~{}) e.g. \~{}\~{}2tildes\~{}\~{} results a line through the text.

\subsection{Links}
A link can be as simple, where just the URL is written 
\begin{verbatim}
<http://www.gnu.org>
\end{verbatim}
A piece of text can be set as a link to a URL
\begin{verbatim}
[Pelican](http://blog.getpelican.com/)
\end{verbatim}
You can add ALT text to it, in the case below Get Pelican would display when your mouse hovers over the link Pelican.
\begin{verbatim}
[Pelican](http://blog.getpelican.com/ "Get Pelican")
\end{verbatim}
\subsection{Lists}
\subsubsection{Unordered Lists}
You can use +, - or * to an unordered list
\begin{verbatim}
* First Item
* Second Item
* Third Item
\end{verbatim}
\begin{verbatim}
- First Item
- Second Item
- Third Item
\end{verbatim}
\begin{verbatim}
+ First Item
+ Second Item
+ Third Item
\end{verbatim}
\subsubsection{Ordered Lists}
\begin{verbatim}
1 First Item
2 Second Item
3 Third Item
\end{verbatim}

\subsection{tables}

\begin{verbatim}
| A     | B     | Result |
| ----- | ----- |:------:|
| True  | True  | True   |
| True  | False | False  |
| False | True  | False  |
| False | False | False  |

\end{verbatim}
\subsection{Inserting an Image}
Inserting an image with markdown.
\begin{verbatim}
![Pelican](../../images/pelican.jpg)
\end{verbatim}
In this case i have an image in a directory in the main file so to be accessed from a markdown file in the /content/pages/ directory you have to specifiy the page


\newpage
\section{Conclusion}
To me the benefits of using a static website generator, are speed of a static website,
no time delay due to interaction with a database or dynamic page generation, great power to customise the site with editing the themes and editing the css. It does take the compostion of content away from concerns of how it is displayed on the page.
Also it does provide something different from the ubiqutious wordpress CMS websites which seem about.
The negatives are that it does require some technical nous to set up, its graphical side may be limited to what is available and your own graphic knowhow, it doenst have the dynamic functionality which plugins provide for many CMS systems.
Overall I like SSG and I think they are a powerful tool to generate a website.

\newpage
\begin{thebibliography}{9}

\bibitem{IBMHYDE} Uche Ogbuji, \emph{Build rapid and lightweigh websites}, IBM Developer Works (February 2013), http://www.ibm.com/developerworks/library/wa-hyde/wa-hyde-pdf.pdf.  
%\bibitem{SBPCIB} A. Weckler, \emph{Social Networking Tools},  Computers in Business : Sunday Business Post (September 2007), 8-11 available at http://www.thepost.ie.  
\end{thebibliography}



\newpage
\appendix
\section{Appendix 1}
\subsection{sitemap.xml}

\copyright 2014, 2015 Conor Gilmer  all rights UNreserved.

\end{document}
